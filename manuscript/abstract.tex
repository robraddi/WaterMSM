% !TEX root = ./main.tex

\section{Abstract}

In this work, we investigate the role water molecules have in the
binding reaction of p53 transactivation domain to the hydrophobic pocket
of MDM2.~ MDM2 provides negative regulation of the tumor suppressor p53
by binding to the intrinsically disordered N-terminal transactivation
domain (TAD), initiating ubiquitination, and ultimately degradation, of
p53.~ Previously, our group generated 831 µs of explicit-solvent
aggregate molecular simulation trajectory data for the MDM2-p53 peptide
binding reaction using large-scale distributed computing, and
subsequently built a Markov State Model (MSM) of the binding reaction \cite{zhou2017bridging}.
 Upon obtaining solvent features, the MSM construction reveals the
slowest motions in de-wetting, which coincide with the previously
acquired conformational degrees of freedom. The solvent shells
contributing most to the first eigenvector of the time-lagged
correlation matrix (along tIC\textsubscript{1}) are centered on T18 of
p53 and H73 of MDM2, in the ranges of (3-6) Å and (0-3) Å, respectively.
In atomic detail, trajectories of 'important' solvent shells were traced
to present a visual representation of these associated water bundles
departing the pocket at discrete occasions.~ Averaging instantaneous
water density over all snapshots for tIC1 and tIC2 clarifies binding
mechanism as well as reaffirming the hydrophobic effect.

