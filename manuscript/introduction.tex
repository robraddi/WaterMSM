% !TEX root = ./main.tex

\section{Introduction}


Tumor suppressor protein p53 is regulated by the E3 ubiquitin-protein
ligase MDM2, which binds the transactivation domain (TAD) of p53 and
recruits it for degradation. Whereas p53 TAD is intrinsically
disordered, it forms a helix upon binding to MDM2.

There has been great interest in using molecular simulation to probe the
mechanisms for this association. Molecular dynamics (MD) has become a
vital tool and is especially useful for understanding the binding
mechanism of proteins. Many MD simulations have previously been used to
study the protein-protein interaction of the MDM2-p53 system, and the
focal point of these results suggest an 'induced-fit', or `fly-casting'
mechanism, in which binding of p53 to MDM precedes folding to a
well-structured helix.

In all of the previously published work on the MDM2-p53 system, there
has been surprisingly subtle insight on the role water plays. It has
been suggested from Marie-Claire et al \cite{bellissent2016water} and many others \cite{spyrakis2017roles,yang2013approaches} that
solvent and protein motions may be interconnected, and the dynamics of
the protein is ``slaved'' to the solvation layers. It could be argued
that protein function, conformational dynamics and stability are all a
result of solvent dynamics. The main driving force underlying protein
folding and binding is the hydrophobic effect, which is mediated by
water. Indeed, explicit-solvent molecular simulations are typically
dominated by large numbers of water molecules.

Normally, molecular simulation studies investigate protein
conformational features, where hydrophobic interactions dictate the
slowest motions. To what extent would it possible to examine protein
binding dynamics using only solvent degrees of freedom? In this work,
our goal is to determine what role water plays in the protein binding
mechanism, particularly in the binding pocket of MDM2 when p53 binds.

To uncover the specificities of water-protein interactions, here we
explore methods to transform the positions of solvent molecules in
molecular simulations to quantifiable features that can be used to build
Markov State Models (MSMs) of conformational dynamics. A Markov State
Model is a network of conformational states and transition rates
connecting them \cite{schwantes2014perspective,bowman2013introduction}. In practice, an MSM is constructed by
estimating a transition probability matrix from the trajectory data,
which determines the likelihood of transitioning from one state to
another in some small time interval.

A crucial step in constructing an MSM is discovering the relevant
metastable states, which for large trajectory datasets requires
conformational clustering, typically done concurrent with a dimensionality
reduction method such as time-independent component analysis (tICA).
tICA is a statistical model that performs a dimensionality reduction by
eigendecomposition of the time-lagged correlation matrix,
\emph{C}\textsuperscript{∆\emph{t}}, which finds the coordinate vectors
along which conformational motions decorrelate most slowly. \cite{bowman2013introduction}

Below, we describe the process of building an MSM of the p53-MDM2
binding reaction using only solvent coordinates, from a previously
published trajectory dataset \cite{zhou2017bridging} of more than 900 µs of aggregate
simulation time in explicit solvent. Thus, our work offers an
unprecedented opportunity to examine the role of solvent in
protein-protein binding.

