% !TEX root = ./main.tex

\section{Discussion}

%Take action to prevent criticism {]}}

%\textbf{{[} Talk about any filled gaps, and how this is relevant to the field. {]} }

%\textbf{If TRP is so important to these trajectories, why don't we see it in the solvent shell analysis as more prominent?}

TRP is found in the conformational analysis, as for the solvent
analysis, we know it is important, but there are residues that correlate
to solvent dynamics with greater magnitude than TRP.

\textbf{Did you use any other metric besides strictly protein-protein
degrees of freedom or strictly a solvent shell features?}

\textbf{Is there a need to use a protein-solvent feature union metric?}

Information of both metrics are very similar. It is unnecessary to
perform a computation with a union of protein-solvent features as the
comparison of solvent only and protein only features reveal almost
identical tICA landscapes. Building markov state models for both the
protein and solvent MSMs with the same parameters show that the
differences between the two are strictly in the slowest motions, where
the landscape from the solvent model that tIC1 correlates to the slowest
de-wetting motions. Contrary to the solvent model, the protein model
uncovers the slowest protein motions where Trp of p53 is very important.

\textbf{How does this method fair?}

Building a WetMSM from Solvent Shells Featurization enables
instantaneous solvent density throughout all data. It is a powerful tool
that uncovers the slowest motions of the system involved in de-wetting.
Although it is a powerful tool, the computational cost of considering
water molecules is much larger than that of the use of strictly protein
indices. Many follow the notion of hydrophobic interactions being the
major influence of conformational change protein binding. Normally,
water is removed after simulations to reduce the computational cost of
analysis. From this analysis, water does indeed effect protein dynamics,
but after revealing that the tICA landscapes are in fact so similar, it
would be advantageous when analyzing protein movement to revert back to
a metric like protein-protein pair distances. Furthermore, if one were
to attempt to analyze solvent or ions within a system, one could uncover
substantial detail when turning to this solvent centric point of view.
However, studying water molecules around the binding site or
protein-protein interface requires attention to detail, the overall
influence shouldn't be snubbed.

Creating water MSMs is nice and you get a lot of information because its
heavily detailed. Information regarding flux of water is pertinent to
drug design, for example, requires the fastidious analysis of water and
ions within certain regions.







